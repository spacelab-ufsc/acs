%
% geomagnetic-field-model.tex
%
% Copyright (C) 2020 by SpaceLab.
%
%
% This work is licensed under the Creative Commons Attribution-ShareAlike 4.0
% International License. To view a copy of this license,
% visit http://creativecommons.org/licenses/by-sa/4.0/.
%
%
% \author Gabriel Mariano Marcelino <gabriel.mm8@gmail.com>
%
% \institution Universidade Federal de Santa Catarina (UFSC)
%
% \version 0.1.0
%
% \date 2020/07/16
%

\chapter{Attitude Representation} \label{ch:attitude-representation}

\section{Introduction}

\paragraph{}
\indent 
	Before building a concise attitude representation, there are two main topics that must discussed: the frames of reference that are going to be used in the computations and the chosen parameterization for spatial rotations. 

\indent
	Specifying an orientation in space only makes sense in a relative perspective, hence the objetive of this section is to define a procedure through which the numbers resulting from attitude computation can be traced back to it's physical interpretation. 

\section{Reference Frames}

\paragraph{}
\indent
	To mathematically describe the satellite's attitude in space, it is convenient to set an inertial reference frame, that is, a reference frame that does not, or at least approximatly not, experience acceleration, mostly because it helps at keeping the mathematics simple and clear. 

\indent 
	In this work, based on what has been presented on \cite{AAUSAT3}, the chosen inertial reference frame is the Earth Centered Inertial Reference Frame. In this reference frame, the x-axis is defined by a unit vector pointing in the direction of the vernal equinox, the point of intersection between the equatorial plane and the eclipitc in which the Sun's trajectory crosses the Equator moving from the Southern Hemisphere to the Northern Hemisphere, as viewed by an observert at Earth. The z-axis  is given by a unit vector in the direction of the geographic north pole axis, and, finally, the y-axis unit vector is the result of the cross-product between the x-axis unit vector and the z-axis unit vector. All equations describing attitude dynamics will be written with respect to this reference frame.

\indent
	Next, an idea of how to the describe attitude based on the definition of an inertial frame will be developed. First, it is relevant to point out that the geometry of the satellite is understood mathematically as a collection of points that move with a moving reference frame attached to the satellite, that is, those points are stationary when observed from such a moving reference frame. This is intuitive. In the inside of a car, if a person is holding a cellphone it seems's like the cellphone is static from the point of view of another passenger, but it is clear that the cellphone's position is changing according to the car's motion, from the perspective of an observer standing outside of the car. It is tempting to the describe, then, a point contained in the satellite's set of points, by:

\begin{equation}
	r^P_I = r^{B}_I + r^P_{B}
\end{equation}

\indent
	This equation reads: "the position of the point P with respect to the inertial reference frame, I, equals the position of the origin of the B reference frame with respect to the inertial reference frame, I, plus the position of the point P with respect to the B reference frame". Where B is a reference frame attached to the satellite. To clearify what this expression means, an intermediate reference frame, B', is defined as a reference frame whose origin coincides with the origin of the B reference frame, but has all of it's axis parallel to the inertial frame's axis, that is, it only differs from the inertial reference frame by a translation operation. This means that the relation between the reference frames B and B' is a rotation operation. Mathematically: 

\begin{equation}
	r^P_I = r^{B}_I + R ^{B'}_{B} r^P_{B'}
\end{equation}

\indent
	The capital letter $R$ is a rotation matrix that transforms the basis vectors of the B' reference frame into the basis vectors of the B reference frame and is the proposed definition of attitude. Attitude is the orientation of a body with respect to some reference placement in space that can be described by a the rotation that must take place in order to make the current placement of the body in space equal to the reference placement. In this case, the reference placement is the inertial reference frame, after the translation required by the position description, and the current placement is the so called Body Reference Frame, a reference frame that moves with the satellite and has two additional special properties. The first special property of the body reference frame is that it's origin is at the satellite's center of mass, making it invariant to rotational movements of the satellite. The second special property of the body reference frame is that in this reference frame the Inertia Tensor is constant and diagonal, which means that the basis vectors for this frame of reference are the normalized eigenvectors of the inertia tensor, that is, the body's principal axes of rotation.

\section{Attitude Parameterization}

\paragraph{}
\indent
	Now that the question "what is attitude?" has been answered, another question must be raised: "how to describe attitude mathematically?". 

\indent
	The straightforward method to construct an attitude representation is to make use of sequential rotations around different orthogonal axes. Using this approach the rotation matrix assumes the form:

\begin{align}
	R_{\phi \psi \ \theta} &= R_{\phi}R_{\psi}R_{\theta} \\
	R_{\phi \psi \ \theta} &= \begin{bmatrix}  1 & 0 & 0 \\ 0  & cos\phi  & -sin\phi \\  0 & sin \phi & cos \phi \end{bmatrix} \begin{bmatrix}  cos\psi & 0 & sin\psi \\ 0  & 1  & 0 \\  -sin \psi & 0 & cos \psi \end{bmatrix} \begin{bmatrix}  cos \theta & -sin \theta & 0 \\ sin \theta  & cos\theta  & 0 \\  0 & 0 & 1 \end{bmatrix}\\
	R_{\phi \psi \ \theta} &= \begin{bmatrix}  cos\psi cos\theta & -cos\psi sin\theta & sin\psi \\ cos\phi sin\theta + sin\phi cos\theta sin\phi & cos \phi cos \theta - sin \phi sin\psi sin\theta & -sin\phi cos\psi \\ sin\phi sin\theta - cos\phi sin\psi cos\theta & cos\theta sin\phi + cos\phi sin\psi sin\theta & cos\phi cos\psi    \end{bmatrix}
\end{align}

\indent 
	Note that the choice of rotation matrices and it's sequence lead to different representations for attitude, in this case the chosen representation is the 3-2-1 sequence, that is, first a rotation about the z-axis, then a rotation about the y-axis and finally a rotation about the z-axis. Also note that this matrices are defined in the B' coordinate system. This is not a very practical representation, mostly beause it is hard to build a simple mathematical description of attitude dynamics based on this representation and it is not a convenient representation for digital implementations of control algorithms, since trigonometric function evaluation can be computationallly expensive and there are several redundant parameters in this representation, a statement supported by the fact that the knowledge of the 3 angles $\phi$, $\psi$, $\theta$, uniquely defines the rotation operation, for a given sequence of rotations, but the rotation matrix has 9 elements.

\indent
	A more elegant descritption of rotations is build upon Euler's Rotation Theorem, for which a proof can be found in \cite{PALAIS}, stated originally as: 

\paragraph{Theorem: } \textit{In whatever way a sphere is turned about its centre, it is always possible to assign a diameter, whose direction in the translated state agrees with that of the initial state.}

\indent
	This means that for every rotation there exists an axis, that is, for every rotation there exists a vector invariant to this rotation. This can be expressed by: 

\begin{equation}
	Rv = v
\end{equation}

\indent
	The theorem's proof states that there exists only one eigenvector for each rotation matrix whose corresponding eigenvalue is real and equal to 1. This vector defines the rotation's unique axis associated with its corresponding matrix, hence any such operation can actually be represented by a rotation axis and an angle of rotation. Based on such an ideia, the Rodrigue's Rotation Formula is going to be derived next to show an alternative way to compute the result of a rotation operation acting upon a vector.

\indent
	Initially, any vector can be decomposed in two components, one parallel to the rotation axis and one perpendicular to it, using the equations: 

\begin{align}
	\vec{x} &= \vec{x}_{||} + \vec{x}_{\_|\_} \\
	\vec{x}_{||} &= (\vec{x} \cdot \vec{e}) \vec{e} \\
	\vec{x}_{\_|\_} &= \vec{x} - (\vec{x} \cdot \vec{e}) \vec{e} \\
\end{align}

\indent 
	Where $\vec{e}$ is the vector representing the rotation axis. The rotation of $\vec{x}$ is, then, the sum of the rotation of its components, due to linearity of rotations, but since the component parallel to the rotation axis inherits the property of invariance to this specific rotation operation, the only component that is actually affected by the operation is the perpendicular component. Note that, once it is given that the rotation only affects vectors perpendicular to the rotation axis, it is also defined that the rotations can be analysed in terms of a basis for the vector space defined by the plane perpendicular to the rotation axis. A convenient basis for such representation is given by the set $\left( \vec{x}_{\_|\_},  \vec{e} \times \vec{x}_{\_|\_} \right)$, constructed using the perpendicular component of the original vector (before rotation) and a second vector that is perpendicular to both the rotation axis, hence belongs to the rotation plane, and to the perpendicular component of the original vector, providing an orthogonal basis for the rotation plane. Now any vector that is the result of a rotation around the rotation axis can be described as: 

\begin{equation}
	\vec{x}' = \vec{x}_{||} + \vec{x}_{\_|\_} cos \alpha + \vec{e} \times \vec{v}sin \alpha
\end{equation}
\indent
	The greek letter $\alpha$ being the rotation angle. Now, given a rotation axis and an angle, the resulting rotated vector can be computed without the usage of rotation matrices. For the last part of this analysis, a final form of attitude parameterization will be shown. This form is the result of some mathematical manipulation of the Rodrigue's rotation formula that results in the following experssion:

\begin{equation}
	\vec{x}' =  (cos \frac{\alpha }{2}-+\vec{e} sin\frac{\alpha }{2}) \vec{x} (cos \frac{\alpha }{2} - \vec{e} sin \frac{\alpha }{2})
\end{equation}

	The careful reader must have realized by now that equation 4.12 implicitly defines a multiplication of the vector $\vec{x}$ by the mathematical entity $(cos \frac{\alpha }{2} \pm \vec{e} sin \frac{\alpha }{2})$. This leads to the introduction of a mathematical object known as a quaternion. The quaternion can be understood as an extension of the complex numbers to a four dimensional space. It is defined by a scalar part and a imaginary part containing three components, written as: 

\begin{equation}
	q = q_0 + iq_1 + jq_2 + kq_3
\end{equation}

\indent
	It is clear that the space of quaternions is isomorphic to the space of four dimensional vectors with real components. The relevant aspect of quaternions is that they come with the definition of a multiplication operation, defined by the multiplication of its basis elements as: 

\begin{align}
	i^2 &= j^2 = k^2 = ijk = -1 \\
	jk &= i \\
	ki & = j \\
	kj & = -i \\
	ik &= -j \\
	ij &= k \\
	ji &= -k\\
\end{align}

\indent
	Multiplication between quaternions can be carried out by using the usual distributive rules from elementary algebra. 

\indent
	To proceed with the discussion about the rotation of vectors in three dimensional space based on quaternions, a pure quaternion must be defined. A pure quaternion is a quaternion whose scalar part is equal to zero. Note that the space of pure quaternions is isomorphic to the space of three dimensional vectors with real components, hence any vector in three dimensional space can also be represented by a pure quaternion. To define quaternion based rotations of three dimensional vectors, inspired by the rotation operations in the complex plane based on multiplication, one must ask if there is a general procedure to multiply a pure quaternion by an arbitrary quaternion and obtain a pure quaternion. The answer is: no. But, if one evaluates the multiplication of three quaternions: 

\begin{equation}
	v' = qvp
\end{equation}

\indent
	Where $v$ is a pure quaternion, it can be shown that if $p$ equals the conjugate of $q$, then $v'$ is always a pure quaternion. The conjugate of $q$ is a quaternion with the same scalar part of $q$ but its imaginary part equals the imaginary part of $q$ multiplied by minus one. By comparing equation 4.22 with equation 4.12, the link between quaternion multiplication and the rotation of vectors in thee dimensional space can finally be established. If

\begin{equation}
	q = cos\frac{\alpha}{2} + (ie_1 + je_2 + ke_3)sin\frac{\alpha}{2} 
\end{equation}

\indent
	Where $e_i$ is the i'th component of the unit vector that defines the rotation axis and $\alpha$ is the rotation angle, then the quaternion operation defined in equation 4.22 represents a rotation around an axis $\vec{e}$ by an angle $\alpha$ and any unit quaternion respecting the structure defined in equation 4.23 is a valid attitude parameterization.

\indent
	Quaternions are going to be used to represent attitude throughout this work.